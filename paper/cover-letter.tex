\documentclass{letter}

\address{University of Birmingham\\School of Physics and
  Astronomy\\Birmingham\\B15 2TT\\United
  Kingdom\\\\\texttt{w.farr@bham.ac.uk}\\+44 783 115 3237}

\signature{Will Farr}

\begin{document}

\begin{letter}{Dr. Leslie Sage\\The Macmillan Building\\4 Crinan Street\\London N1 9XW\\United Kingdom}

  \opening{Dear Dr. Sage:}

  Please find enclosed a submission for Nature Letters entitled ``The
  Occurrence of Earth-Like Planets Around Other Stars.''  The
  submission comprises approximately 1900 words of text and four
  figures (with captions).  We do not at this time have any extended
  sections, supplementary information or supporting manuscripts.  The
  requested brief paragraphs justifying publication in Nature and
  summarising the letter's appeal to a broad audience follow.

  \textbf{For the editor:} The quantity $\eta_\oplus$ is of broad
  scientific interest.  Its estimation is a chief goal of NASA's
  Kepler mission, but is complicated by the difficulty of detecting
  Earth-like planets about other stars.  In addition to constraining
  $\eta_\oplus$ using the complete Kepler data set, we have modelled
  the distribution of planet orbital periods and radii and find that
  it is well-fit by a single Gaussian component.  Our approach to
  reconstructing the period-radius distribution of planets is
  generally applicable to any survey with significant selection
  effects.

  \textbf{For the public:} We use the latest results from NASA's
  Kepler mission to count how many planets of roughly Earth's size
  orbit stars like the Sun in our galaxy at about the same distance
  from their host stars as Earth.  We find that about 4\% of stars
  have a planet the size of Earth in an Earth-like orbit; with about
  100 billion Sun-like stars in the galaxy, this means there are over
  4 billion Earth-sized planets in one year orbits in our galaxy!  

  We thank you for your consideration.

  \closing{Sincerely,}

  
\end{letter}

\end{document}
