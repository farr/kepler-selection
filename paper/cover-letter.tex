\documentclass{letter}

\address{University of Birmingham\\School of Physics and
  Astronomy\\Birmingham\\B15 2TT\\United
  Kingdom\\\\\texttt{w.farr@bham.ac.uk}\\+44 783 115 3237}

\signature{Will Farr}

\begin{document}

\begin{letter}{Dr. Maria Cruz\\Science/AAAS\\1200 New York Avenue NW\\Washington DC 20005\\United States}

  \opening{Dear Dr. Cruz:}

  Please find enclosed a report for submission to Science entitled
  ``The Occurrence of Earth-Like Planets Around Other Stars.''  The
  submission comprises approximately 2500 words of text, including
  four figures (with captions).

  The main result of our paper is a measurement of $\eta_\oplus$, the
  number density of Earth-sized planets in Earth-like orbits about
  stars similar to the sun.  Our measurement uses the complete data
  release from the final quarter of operation of the Kepler sattelite
  (Q17).  The quantity $\eta_\oplus$ is of broad scientific interest.
  Its estimation is a chief goal of NASA's Kepler mission, but is
  complicated by the difficulty of detecting Earth-like planets about
  other stars (i.e.\ selection effects); we deal with this difficulty
  using an innovative statistical technique that simultaneously fits
  the intrinsic distribution of planet periods and radii and a
  parameterised model of the Kepler selection function.  In addition
  to constraining $\eta_\oplus$, we have modelled the distribution of
  planet orbital periods and radii and find that it is well-fit by a
  single Gaussian component.  Our approach to reconstructing the
  intrinsic distribution of planet periods and radii is generally
  applicable to any survey with significant selection effects.

  We thank you for your consideration.

  \closing{Sincerely,}

  
\end{letter}

\end{document}
