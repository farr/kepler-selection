\documentclass{nature}

\usepackage{hyperref}
\usepackage{amsmath}
\usepackage{amssymb}
\usepackage{xcolor}
\usepackage{graphicx}

\bibliographystyle{naturemag}

\newcommand{\etaearth}{\eta_\oplus}
\newcommand{\Rpeak}{R_\mathrm{peak}}
\newcommand{\REarth}{R_\oplus}
\newcommand{\RSun}{R_\odot}
\newcommand{\RStar}{R_\mathrm{star}}
\newcommand{\MStar}{M_\mathrm{star}}
\newcommand{\MSun}{M_\odot}
\newcommand{\Rpl}{\Lambda_\mathrm{pl}}
\newcommand{\Npl}{N_\mathrm{pl}}

\newcommand{\Rbg}{\Lambda_\mathrm{bg}}
\newcommand{\Nbg}{N_\mathrm{bg}}

\newcommand{\rhomin}{\rho_\mathrm{min}}
\newcommand{\rhomax}{\rho_\mathrm{max}}

\newcommand{\email}[1]{\href{mailto:#1}{\nolinkurl{#1}}}

\newcommand{\Will}[1]{\textcolor{cyan}{Will: #1}}
\newcommand{\ilya}[1]{\textcolor{red}{#1}}

\newcommand{\aj}{The Astronomical Journal}
\newcommand{\aap}{Astronomy \& Astrophysics}
\newcommand{\apj}{The Astrophysical Journal}
\newcommand{\apjl}{The Astrophysical Journal Letters}
\newcommand{\apjs}{The Astrophysical Journal Supplement}
\newcommand{\mnras}{Monthly Notices of the Royal Astronomical Society}
\newcommand{\pasp}{Publications of the Astronomical Society of the Pacific}

\newcommand{\earange}{3.9_{-1.6}^{+2.2}\%}
\newcommand{\rplrange}{3.83_{-0.62}^{+0.76}}
\newcommand{\rpeakrange}{1.25_{-0.17}^{+0.16}}
\newcommand{\corrcoeffrange}{0.334_{-0.053}^{+0.052}}
\newcommand{\fposrange}{7.8_{-1.3}^{+1.4}\%}
\newcommand{\ppeakrange}{0.075_{-0.006}^{+0.007}}
\newcommand{\rhominrange}{5.46^{+0.18}_{-0.18}}
\newcommand{\rhomaxrange}{18.8^{+1.9}_{-1.9}}


\begin{document}
\title{The Occurrence of Earth-Like Planets Around Other Stars}

\author{Will M. Farr$^{1}$, Ilya Mandel$^{1}$, Chris Aldridge$^{1}$ \& Kirsty Stroud$^{1}$}

\maketitle

\begin{affiliations}
\item School of Physics and Astronomy, University of Birmingham, Birmingham, B15 2TT, United Kingdom
\end{affiliations}

\begin{abstract}
  The quantity $\etaearth$, the number density of planets per star per
  logarithmic planetary radius per logarithmic orbital period at one
  Earth radius and one year period, describes the occurrence of
  Earth-like extrasolar planets.  Measurement of $\etaearth$ is
  complicated by the difficulty of detecting Earth-like planets in
  Earth-like orbits about Sun-like stars.  Previous
  estimates\cite{Catanzarite2011,Traub2012,Dong2013,Petigura2013,Foreman-Mackey2014}
  place $1\% \lesssim \etaearth \lesssim 34\%$.  These works dealt
  with the problem of selection effects in the sample by either
  analyzing a region of the period-radius parameter space where
  observations are complete and extrapolating to $R = \REarth$ and $P
  = 1 \mathrm{yr}$\cite{Catanzarite2011,Traub2012}, applying a binned
  analysis incorporating survey incompleteness in the period-radius
  plane\cite{Dong2013,Petigura2013} or analysing the results of a
  customised planet detection pipeline on a subset of the Kepler
  observations\cite{Petigura2013,Foreman-Mackey2014}.  Here we present
  a measurement of $\etaearth$ from a parameterised forward model of the
  (correlated) period-radius distribution and the observational
  selection function in the most recent (Q17) data release from the
  Kepler satellite\cite{Borucki2010,Borucki2011,Batalha2013}.  Our
  data set comprises 181,568 systems observed under the Kepler
  exoplanet observing program (mostly G-type stars on the main
  sequence\cite{Batalha2010}), producing 2598 planetary candidates.
  We parameterise the distribution of planetary periods and radii
  using a single, correlated Gaussian component; treat selection
  effects using a parameterised transit detection probability based on
  the measured noise level and stellar properties in the Kepler
  catalog; and include an empirically-parameterised, independent
  component in the period-radius distribution to represent
  false-positive planet detections.  Using our model we can
  simultaneously estimate $\etaearth$, place constraints on the planet
  period-radius distribution function, and determine the degree of
  contamination by false-positive candidate identifications.  We find
  $\etaearth = \earange$ (90\% CL).  We conclude that each
  star hosts $\rplrange$ planets with $P \lesssim 3 \mathrm{yr}$ and
  $R \gtrsim 0.2 \REarth$, that the peak of the planet radius
  distribution lies at $\Rpeak = \rpeakrange \REarth$, and that $\ln
  P$ and $\ln R$ are correlated with correlation coefficient $r =
  \corrcoeffrange$ (all 90\% CL).  Our empirical model for
  false-positive contamination is consistent with the dominant source
  being background eclipsing binary stars\cite{Fressin2013}, with
  $\fposrange$ (90\% CL) of the candidates being false-positives.  The
  distribution of planets we infer could be the result of a
  highly-stochastic planet formation process producing many correlated,
  fractional changes in planet sizes and orbits.  Our approach of
  determining both the intrinsic distribution of objects and selection
  effects empirically from survey data is generally applicable.
\end{abstract}

The Kepler satellite detects planets by observing a decrement in the
photometric intensity of a planet's host star as the planet transits
between the telescope and the star.  The Q17 data release describes
2598 ``candidate'' planetary transit signals identified by the Kepler
team from observations of stars in the ``EX'' observing program (which
are primarily G-type main-sequence stars similar to our own
Sun\cite{Batalha2010}), giving the inferred planetary period and
radius for each.  The fractional depth of a planetary transit signal
depends only on the radii of the planet and its host star.  The signal
to noise ratio of a series of transits about a particular star in the
Kepler satellite scales with planetary period and radius
as\cite{Chatterjee2012}
\begin{equation}
  \rho = \rho_0 \left( \frac{R}{\REarth} \right)^2 \left(
  \frac{P}{1\,\mathrm{yr}} \right)^{-1/3},
\end{equation}
where $\rho_0$ is the signal to noise ratio of a Earth-radius planet
in a one-year orbit about that star, which depends on the number of
quarters of observation of that star, the stellar radius and mass, and
the intrinsic variability of the stellar
intensity\cite{Christiansen2012}.  In our analysis, we obtain these
quantities from the Kepler Input Catalog\cite{Batalha2010,Brown2011}
and the MAST Kepler
archive\footnote{\url{http://archive.stsci.edu/kepler/}}.  To a good
approximation (see Fig.\ \ref{fig:selection} below), the detectability
of a series of planetary transits in the Kepler data set is a function
of the signal to noise ratio of the series.  Because the detectability
of planet transits depends on both period and radius, it is important
to consider the joint (i.e., two-dimensional) distribution of these
quantities in the data\cite{Tabachnik2002,Youdin2011}.  We model the
detection probability of a transit as a function that rises linearly
in the log of the signal to noise ratio from zero at a threshold
signal to noise to one at a larger signal to noise:
\begin{equation}
  \label{eq:pdetect}
  p_\mathrm{detect} = \begin{cases}
    0 & \rho < \rhomin \\
    \frac{\log \rho - \log \rhomin}{\log \rhomax - \log \rhomin} &
    \rhomin < \rho < \rhomax \\
    1 & \rhomax < \rho
  \end{cases},
\end{equation}
where $\rhomin$ and $\rhomax$ are parameters of our model.  We find
$\rhomin = \rhominrange$ and $\rhomax = \rhomaxrange$ (90\% CL), in
rough agreement with Refs.\ \cite{Borucki2011,Batalha2013}.  A plot of
our inferred detection probability appears in Fig.\ \ref{fig:det-bg}

The probability that a planet's orbital plane will align with the line-of-sight to
Earth and thereby produce a transit signal is
\begin{equation}
  \label{eq:ptransit}
  p_\mathrm{transit} = 0.0016\, \frac{\RStar}{\RSun}
  \left(\frac{\MStar}{\MSun}\right)^{-1/3} \left(\frac{P}{1\,\mathrm{yr}}\right)^{-2/3}.
\end{equation}
Putting Eq.\ \ref{eq:pdetect} and \ref{eq:ptransit} together, the
probability that Kepler will detect a planet of radius $R$ orbiting
its host star at period $P$ is 
\begin{equation}
  p_\mathrm{select} = p_\mathrm{transit} p_\mathrm{detect}.
\end{equation}

A correlated log-normal distribution of planets in period and radius
would be a natural outcome of a stochastic planet formation process
that produced many correlated, fractional changes in planet sizes and
orbits.  As we shall see (Figure\ \ref{fig:selection}), this simple
model combined with the aforementioned selection function, provides a
good fit to the Kepler candidate distribution.  In our model, observed
planets populate the candidate $P$-$R$ plane with number density
\begin{equation}
  \label{eq:foreground-rate}
  \frac{dN_\mathrm{obs}}{d\ln P\, d\ln R} = \left[ \sum_\mathrm{stars}
    p_\mathrm{select}(P, R) \right] \Rpl N\left[ \mu, \Sigma
    \right]\left( \ln P, \ln R \right),
\end{equation}
where $\Rpl$, $\mu$, and $\Sigma$ are parameters of our model, with
$\Rpl$ the average number of planets per star, $\mu = \left[ \mu_P,
  \mu_R \right]$ the mean of $\ln P$ and $\ln R$, and $\Sigma = \left[
  \left[ \Sigma_{PP}, \Sigma_{PR} \right], \left[ \Sigma_{PR},
    \Sigma_{RR} \right]\right]$ the covariance matrix of $\ln P$ and
$\ln R$; $N\left[ \mu, \Sigma \right](x,y)$ is the normal distribution
Our model assumes that planets appear around their host stars in a
Poisson process; this is almost certainly wrong in
detail\cite{Weissbein2012}, but nevertheless provides a good fit to
the observed data (see Figure \ref{fig:selection}).

In addition to true planetary signals, we model a false-positive
background of planet candidates empirically, assuming they populate
the candidate $P$-$R$ plane with a number density that has a linear
gradient across a rectangular region in the $\ln P$-$\ln R$ plane:
\begin{equation}
  \label{eq:background-rate}
  \frac{d\Nbg}{d \ln P \, d \ln R} = \frac{\Nbg}{\Delta \ln P \Delta
    \ln R} \left( 1 + \vec{\gamma} \cdot \left[ \ln P - \ln P_\mathrm{mid} , \ln
    R - \ln R_\mathrm{mid} \right] \right),
\end{equation}
where $\Delta \ln P = \ln P_\mathrm{max} - \ln P_\mathrm{min}$, $\ln
P_\mathrm{mid} = 1/2\left(\ln P_\mathrm{max} - \ln P_\mathrm{min}
\right)$, $\Delta \ln R = \ln R_\mathrm{max} - \ln R_\mathrm{min}$,
$\ln R_\mathrm{mid} = 1/2\left(\ln R_\mathrm{max} - \ln R_\mathrm{min}
\right)$.  $\Nbg$, the expected number of background false-positive
events; $P_\mathrm{max}$, $P_\mathrm{min}$, $R_\mathrm{max}$, and
$R_\mathrm{min}$, the boundaries in the $P$-$R$ plane within which
background events appear; and $\gamma$, the gradient in the number
density of background events, are parameters of our model.  This is a
purely empirical model for the background contamination, but is
reasonable if the chief contaminant is background eclipsing
binaries\cite{Fressin2013,Duquennoy1991}.

Unlike Ref.\ \cite{Foreman-Mackey2014}, we do not attempt to model the
observational uncertainties in the estimated periods and radii from
the Kepler candidate data set.  In spite of several candidates with
very large uncertainties in measured parameters, we have found that our
fit is essentially unchanged when applied to synthetic observations
with periods and radii re-drawn from the range of observational
uncertainties quoted in the Q17 data release.  

The likelihood of the observed periods and radii under our model is an
inhomogeneous Poisson likelihood\cite{Farr2013,Youdin2011} with a rate
that is the sum of Eq.\ \eqref{eq:foreground-rate} and
Eq.\ \eqref{eq:background-rate}.  We impose priors on our 15 model
parameters as follows: for the planet occurrence rate $\Rpl$ and
(implicitly) the parameters describing selection effects, we impose a
$\frac{1}{\sqrt{\Npl}}$ prior; for the background rate $\Rbg$ we
impose a $\frac{1}{\sqrt{\Rbg}}$ prior; for the selection model
parameters $\rhomin$ and $\rhomax$ we impose a log-normal prior with
unit width at signal to noise ratios  of 3 and 11, respectively; in all other parameters
we impose a flat (i.e., constant-density) prior.  The product of
likelihood and prior gives a Bayesian posterior density function on
the fifteen-dimensional parameter space of our model.  We sample from
this function using the \texttt{emcee}
sampler\cite{Foreman-Mackey2013}.  The posterior describes
simultaneously the intrinsic distribution and number of exoplanets,
the amount and distribution of the contaminating false-positive events
in the candidate data set, and the selection function of the
instrument for true planetary transit events.

The main result of this paper, the posterior distribution for
$\etaearth$, the number density of Earth-like planets, marginalised
over all other parameters in our model (i.e., incorporating our
uncertainty about contamination, selection effects, intrinsic
distribution of planets, etc) appears in Fig.\ \ref{fig:eta-earth}.
Recall that
\begin{equation}
  \etaearth = \left. \frac{dN}{d \ln P \ln R} \right|_{R = \REarth, P
    = 1\,\mathrm{yr}} = \Rpl N\left[ \mu, \Sigma \right]\left( \ln 1\,\mathrm{yr},
  \ln \REarth \right),
\end{equation}
which is roughly the number of planets per star with periods and radii
within a factor of $\sqrt{e}$ of Earth's.  We find $\etaearth =
\earange$ (90\% CL).  Our model also gives an estimate of the number
of planets of any radius and period per star; the posterior for this
quantity, marginalised over all other parameters also appears in
Fig.\ \ref{fig:eta-earth}.  We find $\Rpl = \rplrange$ (90\% CL).

Our model allows us to produce a posterior on the distribution of
planets in the period-radius plane, and the probability that any given
planetary candidate is a planet instead of a background contaminant;
these posteriors appear in Fig. \ref{fig:foreground-dist}.  Our model
finds that the false-positive rate in the candidate data set is
$\fposrange$ (90\% CL), consistent with previous
work\cite{Fressin2013} estimating the contamination in the Kepler
candidate set.  Our model has the peak of the planet period-radius
distribution at $\Rpeak = \rpeakrange \REarth$, $P_\mathrm{peak} =
\ppeakrange \mathrm{yr}$, and the distribution of planetary radii and
periods is correlated, with correlation coefficient $r =
\corrcoeffrange$ (all at 90\% CL).

Our model predicts a distribution for future observed data consistent
with the already-observed candidate set.  These predictions can be
used to perform graphical and posterior-predictive model
checking\cite{Gelman2013}.  Fig.\ \ref{fig:selection} compares the
predictions of our model for observed periods and radii (incorporating
both planetary transits and background events) with the candidate set.
This is a particularly stringent test of our parameterised selection
model since the observed periods and radii are strongly influenced by
the selection function of the Kepler telescope and pipeline.  Except
for the known sub-population of hot
Jupiters\cite{Albrecht2012,Naoz2012}, our model provides a very good
fit to the observed data.  That a simple log-normal distribution in
period and radius fits the observed distribution of planets well may
indicate that planet formation is a stochastic process with many
small, correlated, and multiplicative influences on planet period and
radius resulting, from the central limit theorem, in a log-normal
distribution in these parameters.

The methods and analysed data sets of
Refs.\ \cite{Petigura2013,Foreman-Mackey2014} are most comparable to
ours.  These studies used the same data set,
produced\cite{Petigura2013} from a subset of the available Kepler data
and a customised pipeline to search for transit signals.  They both
accounted for selection effects by measuring the recoverability of
synthetic transit signals injected into their data, in contrast to our
approach of empirically determining them from the observed data.
Neither study attempted to account for contamination from
falsely-identified candidate transit events, controlling this instead
through careful choice of threshold.  Both studies used a more
flexible model for the intrinsic distribution of planets than ours.
Our result for $\etaearth$ is consistent with, but more precise than,
Ref.\ \cite{Foreman-Mackey2014} and (somewhat) inconsistent with
Ref.\ \cite{Petigura2013}.  


\begin{figure}
  \includegraphics[width=\columnwidth]{pars}
  \caption{\label{fig:eta-earth} \textbf{Posteriors on $\etaearth$ and
      $Rpl$ accounting for selection effects and false-positive
      detections.}  (Top) The inferred posterior density on $\etaearth
    = dN/d\ln P \, d\ln R \left( 1\, \textnormal{yr}, R_\oplus
    \right)$.  Vertical lines indicate the 90\% credible range.  We
    find $\etaearth = \earange$.  (Bottom) The inferred posterior on
    $\Rpl$, the number of planets per star with $P \lesssim 3
    \mathrm{yr}$ and $R \gtrsim 0.2 \REarth$.  Vertical lines indicate
    the 90\% credible range.  We find $\Rpl = \rplrange$.}
\end{figure}

\begin{figure}
  \includegraphics[width=\columnwidth]{bg}
  \caption{\label{fig:det-bg} \textbf{Inferred detection probability
      and density of background contamination.} (Top) The inferred
    detection probability versus signal-to-noise ratio (see
    Eq.\ \eqref{eq:pdetect}) from our parameterised model of selection
    effects.  The solid line is the posterior median detection
    probability and the shading gives the 90\% credible posterior
    interval.  Our inferred detection probability is in rough
    agreement with the measurements of detection efficiency in
    Refs.\ \cite{Borucki2011,Batalha2013}.  (Bottom) The number
    density of false-positive candidate signals, $d\Nbg/d\ln P \, d\ln
    R$ (see Eq.\ \eqref{eq:background-rate}).  The density is highest
    at large candidate period and radius, consistent with the dominant
    source of contamination being background eclipsing
    binaries\cite{Fressin2013}.  Overall, our model finds $\fposrange$
    of the candidates are false-positive background signals,
    consistent with the analysis in Ref.\ \cite{Fressin2013}.}
\end{figure}

\begin{figure}
  \includegraphics[width=\columnwidth]{foreground-dist}
  \caption{\label{fig:foreground-dist} \textbf{The inferred planet
      period--radius distribution accounting for selection effects and
      false-positives.}  (Upper Left) The planet number density per
    logarithmic planet radius.  The density peaks at $\Rpeak =
    \rpeakrange \REarth$ (90\% CL).  (Upper Right) The planet number
    density in the period--radius plane.  The inferred correlation
    coefficient between $\ln P$ and $\ln R$ is $r = \corrcoeffrange$.
    (Lower Left) Scatter plot of the radius and period of the Kepler
    planet candidates.  Color indicates the posterior false-positive
    probability for each candidate.  Overall, the model prefers a
    false-positive rate of $\fposrange$ (90\% CL).  The primary
    contaminant is probably background eclipsing binaries; our
    contamination rate is consistent with previous
    work\cite{Fressin2013}. (Lower Right) The planet number density
    per logarithmic planet period.  The density peaks at $P =
    \ppeakrange \mathrm{yr}$ (90\% CL). }
\end{figure}

\begin{figure}
  \includegraphics[width=\columnwidth]{selection}
  \caption{\label{fig:selection} \textbf{Comparison of synthetic data
      sets produced from the forward model incorporating selection
      effects with observed candidates.}  (Upper Left) The observed
    (black curve) and synthetic (blue curve; including planets and
    false positives, and using the fitted selection model to
    down-select the candidates from the planet distribution)
    normalised candidate density per logarithmic radius.  Except for a
    discrepancy at $R \simeq 10 \REarth$---associated with hot
    Jupiters, a distinct planetary
    population\cite{Albrecht2012,Naoz2012}---the model produces a good
    fit to the observed candidates over the range of reported radii.
    Note particularly the tail at large radii that comes from
    background contaminants in both observed and synthetic data.
    (Upper Right) Scatter plot of the observed candidates (black
    circles) and a posterior-averaged draw of observed candidates from
    the model (blue circles).  (Lower Left) Scatter plot of the
    observed candidates.  Colors indicate the posterior-averaged
    selection probability for each planet about its host star.  
    The selection probability is treated a product of a geometric
    factor giving the probability of an isotropically-oriented orbit
    producing a transit and a signal-to-noise-ratio-dependent transit
    detection probability.  (Lower Right) The observed (black curve)
    and synthetic (blue curve; including planets and false positives,
    and using the fitted selection model to down-select the candidates
    from the planet distribution) normalised candidate density per
    logarithmic period.  Except for the aforementioned hot Jupiter
    peak at $P \simeq 1 \mathrm{day}$ the model produces a good fit to
    the observed candidates over the range of reported periods.}
\end{figure}

\bibliography{kepler-selection}

\begin{addendum}
\item The code implementing this analysis is available under an
  open-source ``MIT'' license at
  \url{https://github.com/farr/kepler-selection}.  This work was
  supported by the Science and Technology Facilities Council.
  Computations in this work were performed on the University of
  Birmingham's BlueBEAR cluster.  Some of the data presented in
  this paper were obtained from the Mikulski Archive for Space
  Telescopes (MAST). STScI is operated by the Association of
  Universities for Research in Astronomy, Inc., under NASA contract
  NAS5-26555. Support for MAST for non-HST data is provided by the
  NASA Office of Space Science via grant NNX13AC07G and by other
  grants and contracts.  This paper includes data collected by the
  Kepler mission. Funding for the Kepler mission is provided by the
  NASA Science Mission directorate.
\item [Author Contributions] All authors assisted in the computational
  modelling, discussed the results, and edited the manuscript.
\item [Reprints] Reprints and permissions information is available at
  \url{www.nature.com/reprints}.
\item[Competing Interests] The authors declare that they have no
  competing financial interests.
\item[Correspondence] Correspondence and requests for materials should
  be addressed to W.M.F.\ (email: \email{w.farr@bham.ac.uk}).
\end{addendum}

\end{document}
