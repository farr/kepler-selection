\documentclass{nature}

\usepackage{hyperref}
\usepackage{amsmath}
\usepackage{amssymb}
\usepackage{xcolor}

\bibliographystyle{naturemag}

\newcommand{\etaearth}{\eta_\oplus}
\newcommand{\Rpeak}{R_\mathrm{peak}}
\newcommand{\REarth}{R_\oplus}

\newcommand{\email}[1]{\href{mailto:#1}{\nolinkurl{#1}}}

\newcommand{\Will}[1]{\textcolor{cyan}{Will: #1}}

\newcommand{\apj}{The Astrophysical Journal}
\newcommand{\apjs}{The Astrophysical Journal Supplement}

\begin{document}
\title{The Occurrence of Earth-Like Planets Around Other Stars}

\author{Will M. Farr$^{1}$, Chris Aldridge$^{1}$, Kirsty Stroud$^{1}$ \& Ilya Mandel$^{1}$}

\maketitle

\begin{affiliations}
\item School of Physics and Astronomy, Univeristy of Birmingham, Birmingham, B15 2TT, United Kingdom
\end{affiliations}

\begin{abstract}
  The quantity $\etaearth$, the number density of planets per star per
  logarithmic planetary radius per logarithmic orbital period,
  describes the occurrence of Earth-like extrasolar planets.
  Measurement of $\etaearth$ is complicated by the difficulty of
  detecting Earth-like planets in Earth-like orbits about Sun-like
  stars; previous
  estimates\cite{2011ApJ...738...81W,2011ApJ...738..151C,2012ApJ...745...20T,2013ApJ...778...53D,2013PNAS..11019273P}
  place $1\% \lesssim \etaearth \lesssim 34\%$.  Here we present
  constraints on $\etaearth$ from a parameterised forward model of the
  (correlated) period-radius distribution and the observational
  selection function in the most recent (Q17) data release from the
  Kepler
  satellite\cite{2010Sci...327..977B,2011ApJ...736...19B,2013ApJS..204...24B}.
  We parameterise the intrinsic distribution of planetary periods and
  radii using a single, correlated Gaussian component; treat selection
  effects using a parameterised transit detection probability based on
  the measured noise level and stellar properties in the Kepler
  catalog; and include an empirically-parameterised, independent
  component in the planet period-radius distribution to represent
  false-positive planet detections.  We find $\etaearth =
  4.1^{+2.3}_{-1.7}\%$ (90\% CL).  Additionally, we find that each
  star hosts $4.04_{-0.68}^{+0.85}$ planets with $P \lesssim 3
  \mathrm{yr}$ and $R \gtrsim 0.2 \REarth$ and that the peak of the
  planet radius distribution lies at $\Rpeak = 1.19_{-0.18}^{+0.17}
  \REarth$ (all 90\% CL).  Our empirical model for false-positive
  contamination is consistent with the dominant source being backgroud
  eclipsing binary stars, with $1.1_{-0.18}^{+0.2} \times 10^{-3}$
  false-positives per star.  \Will{More science stuff here.}
\end{abstract}

\bibliography{kepler-selection}

\begin{addendum}
\item WMF and IM are supported by a STFC consolidated grant number
  NNNN.  Computations in this work were performed on the University of
  Birmingham's BlueBEAR cluster.  Some/all of the data presented in
  this paper were obtained from the Mikulski Archive for Space
  Telescopes (MAST). STScI is operated by the Association of
  Universities for Research in Astronomy, Inc., under NASA contract
  NAS5-26555. Support for MAST for non-HST data is provided by the
  NASA Office of Space Science via grant NNX13AC07G and by other
  grants and contracts.  This paper includes data collected by the
  Kepler mission. Funding for the Kepler mission is provided by the
  NASA Science Mission directorate.
\item [Author Contributions] All authors assisted in the computational
  modelling, discussed the results, and edited the manuscript.
\item [Reprints] Reprints and permissions information is available at
  \url{www.nature.com/reprints}.
\item[Competing Interests] The authors declare that they have no
  competing financial interests.
\item[Correspondence] Correspondence and requests for materials should
  be addressed to W.M.F.\ (email: \email{w.farr@bham.ac.uk}).
\end{addendum}

\end{document}
